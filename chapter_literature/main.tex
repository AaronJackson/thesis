\chapter{Literature Review}


This section attempts to provide a comprehensive literature review of
the works relevant to this thesis. This thesis focuses primarily on
the 3D reconstruction of human faces and bodies. Much of the
literature discussed in this chapter has an obvious
relevance. However, in the cases which are less obvious, connections
have attempted to be made with references to the relevant chapter and
section.


\section{Semantic Segmentation}

Semantic segmentation is the process by which pixels of an image are
labelled by the objects contained in the image. For example, all dogs
may be labelled as red, while cats may be labelled as green. In order
for this to work, context is required, both locally and
globally. Global context tells us what is in the image and roughly
where a particular object may be located. Local context then refines
this global information, defining boundaries between classes with
greater precision.

Classical approaches to semantic segmentation involved first
segmenting the image using statistical models, followed by
individually classifying the segmented objects. See for
example~\cite{arbelaez2012semantic,carreira2012semantic}, which use a
Conditional Random Field (CRF) to separate the image. However, these
methods generally struggle to adequately incorporate high and low
level feature descriptors necessary for accurate segmentation and
labelling.

CNNs drastically changed the landscape of semantic segmentation. The
first work producing satisfactory semantic segmentation was the Fully
Convolutional Network (FCN) described in~\cite{long2015fully}, which
is based on VGG-16~\cite{simonyan2014vgg}. In this work, information
is forked from different layers throughout the network. Each fork is
used to produce an intensity map for all known class (initially 59,
targeted at the PASCAL VOC 2010
dataset~\cite{everingham2010pascal}). Forks earlier on in the network
provide very detailed but noisy predictions about the location of
classes, where as the latter forks are have very little noise, but can
only approximate the location of a class. A diagram of the structure
of FCN is shown in Figure~\ref{fig:background:fcn}. FCN was not
intended to be trained from scratch. Instead, training was initialised
from the parameters of VGG-16, once the network had been trained on
the ImageNet dataset~\cite{krizhevsky2012imagenet} for the task of
classification. The parameters in the last few layers of the network
were reshaped from linear layers into a $1\times 1$ convolution. An
interesting side effect of which, is the ability for the network to
accept inputs of arbitrary sizes. The size of the input no longer
dictates the number of parameters in the linear layers. Due to its
simplicity and (at the time) state of the art results, it is no
surprise the ideas behind FCN have become the standard for modern
semantic segmentation methods.

Still, one of the limitations of FCN is that the resolution of its
predictions are quite low. As such, several methods have proposed
extensions to FCN to compensate for this limitation. The most common
addition has been to add a CRF to the output of the network, to
provide further refinement. The work of \cite{chen2015semantic}
first upsamples the predicted scores using bilinear interpolation
and then refines the output by applying a dense CRF. The method of
\cite{zheng2015conditional} performs recurrent end-to-end training of
the FCN and the dense CRF. Finally, the work in \cite{noh2015learning}
employs learnt deconvolution layers, as opposed to fixing the
parameters with an interpolation filter (as in FCN). These filters
learn to reconstruct the object's mask, in addition to upsampling its
input.

In the past year or so, the Encoder-Decode network architecture has
gained a great deal of popularity. This architecture reduces the
spatial dimensionality of the feature as they are passed through the
\textit{encoder}. These features are then upsampled again using a
\textit{decoder} part of the network, while being recombined with
lower level information, much like in FCN~\cite{long2015fully}. This
architecture has also been referred to as an hourglass network. This
name was first used in~\cite{newell2016stacked}, and performed human
pose estimation with great precision. This level of precision is due
to the being suited to combining features from the high and low parts
of the network efficiently (i.e. combining both global and local
context). As such, similar architectures have been used for semantic
segmentation problems. For example,
SegNet~\cite{badrinarayanan2017segnet} uses an Encoder-Decoder
architecture is used to perform semantic segmentation on 11 classes
for autonomous driving in real time.

Architectures aside, a main drawback of this supervised approach to
semantic segmentation is producing large quantities of detailed
groundtruth semantic masks. There are several large datasets already
available, such as PASCAL VOC~\cite{everingham2010pascal}, Microsoft
COCO~\cite{lin2014microsoft} and
Cityscapes~\cite{cordts2016cityscapes}. All of these datasets are
fairly constrained in terms of the number of known
classes. Respectively, these datasets have 59, 183 and 30 known
classes. This is problematic when it comes to making semantic
segmentation a viable tool for general semantic segmentation
\textit{in-the-wild}. As such, numerous approaches have been devised
to train these models in ways requiring less human
intervention. In~\cite{richter2016playing}, hooks are added to
visually detailed computer games, to allow automatic generation of
instance aware semantic segmentation masks. A drawback of this is that
most games are far from realistic enough to work on real images. As
such, models trained on this data will also require domain adaptation
- which is an entire research area in itself. In~\cite{bearman2016s},
a weakly supervised approach is presented, where groundtruth semantic
segmentation masks are avoided entirely. Instead, the method learns
the semantic segmentation task from just a single point on each
object.

Another approach is described in~\cite{hu2018learning}, where the
training set is formed of images where all classes are annotated with
bounding boxes. However, only some classes have groundtruth masks. The
network is trained to produce both bounding boxes and semantic
segmentation masks simultaneously by learning a mapping between the
visual embedding from the bounding box feature extractors, to the
semantic segmentation feature extractors. The paper describes this
working on 3000 segmentation classes. They refer to this approach as
\textit{partially} supervised, as some groundtruth data is
required. Additionally, the method has been shown to be able to
segment up to 3000 classes - far higher than~\cite{long2015fully},
which was trained on the 59 classes from VOC PASCAL
2010~\cite{everingham2010pascal}.

The primary focus of this thesis is not semantic
segmentation. However, in Chapter~\ref{chapter:seg}, we describe our
method for facial part segmentation, in which each component of the
face (such as eyes, lips, nose, etc) are assigned a pixel wise
label. Deep learning based approaches to semantic segmentation,
particularly the fully convolutional approach described in
~\cite{long2015fully}, have been heavily influential in our work on 3D
reconstruction.

\begin{figure}
  \centering
  \begin{tikzpicture}[scale=0.9]
  \networkLayer{3.0}{0.48}{0}{0.0}{color=green!30}{96}
  \networkLayer{1.5}{1.28}{0.48}{0.0}{color=green!30}{256}
  \networkLayer{0.75}{1.92}{1.76}{0.0}{color=green!30}{384}
  \networkLayer{0.75}{1.92}{3.68}{0.0}{color=green!30}{384}
  \networkLayer{0.75}{1.28}{5.60}{0.0}{color=green!30}{256}
  \networkLayer{0.37}{2.05}{6.88}{0.0}{color=green!30}{4096}
  \networkLayer{0.37}{2.05}{8.93}{0.0}{color=green!30}{4096}
  \networkLayer{0.37}{0.5}{10.98}{0.0}{color=green!30}{1000}

  \networkLayer{0.75}{0.5}{12}{0.0}{color=red!30}{$\times 2$}
  \networkLayer{1.5}{0.5}{13}{0.0}{color=red!30}{$\times 2$}
  \networkLayer{3}{0.5}{14.5}{0.0}{color=red!30}{$\times 2$}

  \draw [>->] (1.92,1.75) -- (1.92,2.5) -- (13.5,2.5);
  \draw [>->] (6.88,1) -- (6.88,1.25) -- (12.5,1.25);

  \draw (13.5,3) circle [radius=0.3] node {$+$};
  \draw (12.5,1.75) circle [radius=0.3] node {$+$};
\end{tikzpicture}
\caption[The VGG-16 network]{Basic VGG-16~\cite{simonyan2014vgg}
  network with modifications added from
  FCN~\cite{long2015fully}. Information from lower levels of the
  network summed to upsampled output of the tailing end of the
  network. The number of features is shown below each convolution.}
\label{fig:background:fcn}
\end{figure}

\subsection{Part Segmentation}

\subsection{Facial Part Segmentation}

\section{Face Alignment}

Until recently, the highest performing methods for face alignment were
based on a cascaded regression approach. These methods estimate the
location of facial landmarks by applying a sequence of
regressors. Provided to these regressors is typically
SIFT~\cite{lowe2004distinctive} and other hand selected
features. These regressors are applied in a cascaded fashion where the
input to regressor $k$ is the estimate from regressor $k-1$. More
details of such approaches to face alignment can be found
in~\cite{sanchez16,cao2014facewarehouse,xiongsupervised,zhu2015face,tzimiropoulos2015project}.

The remainder of this section will focus on deep learning based
approaches to face alignment. Many of these methods estimate the
facial landmarks by regressing a 2D Gaussian on top of each landmark
location~\cite{bulat2016two,bulat2017far,mahpod2018facial,kowalski2017deep,merget2018robust}.
This spatial approach is commonly referred to as heatmap
regression. In~\cite{bulat2016two}, the authors propose a method for
3D face alignment by first detecting the 2D landmarks using heatmap
regression. If desired, the process can be stopped here. Optionally,
this stack of heatmaps is passed through a second network, this time
for regression, accompanied by the original input image. The output of
this second stage is 68 $(x, y, z)$ landmarks. This work was extended
in~\cite{bulat2017far}, where the second regressor
in~\cite{bulat2016two} is replaced with a CNN producing a 3D heatmap
representation.

This same heatmap based approach was used in a multi-stage approach
described in~\cite{kowalski2017deep}, where the landmark detections
are refined after each stage. Between each stage, the \textit{current}
landmark prediction and some generated features are passed. At the
beginning of each stage, the inputs are transformed using the
landmarks to make the problem less challenging.

..


Our work does not attempt to improve the \textit{state-of-the-art} in
face alignment. Instead, we optionally employ face alignment as a tool
for guidance in Chapters~\ref{chapter:seg} and~\ref{chapter:face},
which we show can improve the performance of these methods quite
significantly.

% zhang2016joint
% Joint face detection and alignment using multitask cascaded
% convolutional networks

% ren2014face
% Face alignment at 3000 fps via regressing local binary features

% ranjan2017hyperface
% HyperFace: A Deep Multi-task Learning Framework for Face Detection,
% Landmark Localization, Pose Estimation, and Gender Recognition

% mahpod2018facial
% Facial Landmark Point Localization using Coarse-to-Fine Deep Recurrent
% Neural Network

\section{Human Pose Estimation}

\section{Depth Estimation}

A common approach to reconstructing the visible geometry of a 3D
object or scene is to regress a distance value for each pixel from the
camera. From this, a partial 3D reconstruction can be formed, or
refined. In~\cite{saxena2006learning}, a method is presented which
uses a Markvov Random Field (MRF) to regress the depth from a single
image. This method struggles to regress sharp variations in depth,
such as between foreground and far background. This method strongly
influenced~\cite{liu2010single}, which added a second MRF capable of
capturing a more global context using superpixels.

As the popularity of Convolutional Neural Networks (CNNs) grew,
interest moved away from MRF based approaches. There are now numerous
approaches to depth estimation using CNNs.

\section{Face Reconstruction}

\subsection{Shape from Shading}

\subsection{Morphable Model Fitting}


\section{Human Body Reconstruction}











%%% Local Variables:
%%% TeX-master: "../thesis"
%%% End:

