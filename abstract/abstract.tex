

\begin{abstract}
  3D Face reconstruction is the process of estimating the full 3D
  geometry of a human's face from one or more images. Applications of
  3D face reconstruction span many areas, from personalisation of
  video games and trying on accessories online, to measuring emotional
  arousal for psychological studies and in medicine, such as
  simulating the result of reconstructive surgery. Approaches to 3D
  face reconstruction generally depend on a 3D Morphable Model (3DMM)
  - a parametric model, where the shape, pose and expression can be
  adjusted using a small number of parameters. While methods based on
  such techniques can work well on frontal images, they often begin to
  fail on cases of large pose, difficult expression, occlusion, and
  bad lighting. Additionally, encoding detail in so few parameters is
  not possible.

  In this thesis, we propose a novel approach to the problem of 3D
  face reconstruction: \textit{Volumetric Regression Networks}. Our
  non-parametric approach constrains the problem to the spatial domain
  using an end-to-end network which directly regresses the 3D geometry
  using a volumetric representation. This avoids the need for 3DMM
  generation, which involves finding correspondence between all
  vertices of all training samples, but also the fitting stage, which
  requires solving a difficult optimisation problem. We demonstrate
  that doing so can not only provide state-of-the-art results, but
  also be adapted to other deformable objects, such as the full human
  body.
\end{abstract}


%%% Local Variables:
%%% TeX-master: "../thesis"
%%% End: 