\chapter{Conclusion}
\label{chapter:conclusion}


%%%% A summary of the main part of the text

In this thesis, we have proposed a novel approach to 3D face
reconstruction, which can be learnt end-to-end and is resilient to
large pose, unusual expressions and occlusion, while being insensitive
to alignment, minimising the requirements for careful
initialisation. We compared our method to several other
state-of-the-art methods, and out performed all of them by a large
margin. We showed how the related task of face alignment can be used
to assist the 3D reconstruction process. We demonstrated that our
method is able to work well also on the task of human body
reconstruction. In doing so, we also demonstrated that our method is
able to reconstruct fine details, such as wrinkles in clothing, when
trained with high quality data. This strongly suggests that our method
will also be capable of working on other deformable objects.

%%%%    A deduction made on the basis of the main body
%%%%    Your personal opinion on what has been discussed
%%%%    A statement about the limitations of the work
%%%%    A comment about the future based on what has been discussed
%%%%    The implications of the work for future research


\section{Future Work}


\subsection{Incremental Detail Refinement} % Detail

\subsection{Texture Re-synthesis}

\subsection{Fixed Correspondence}




%%% Local Variables:
%%% TeX-master: "../thesis"
%%% End:
