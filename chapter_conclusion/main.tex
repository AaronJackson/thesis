\chapter{Conclusion}
\label{chapter:conclusion}


%%%% A summary of the main part of the text

In this thesis, we have proposed a novel approach to 3D face
reconstruction, which can be learnt in an end-to-end fashion and is
resilient to large pose, unusual expressions and
occlusion. Additionally, our method is insensitive to alignment,
minimising the requirements for careful initialisation, improving
reliability overall. We compared our method to several other
state-of-the-art methods, and out performed all of them by a large
margin. We showed how the related task of face alignment can be used
to assist the 3D reconstruction process. We also demonstrated that our
method is able to work well on the task of human body
reconstruction. In doing so, we also demonstrated that our method is
able to reconstruct fine details, such as wrinkles in clothing, when
trained with high quality data. This strongly suggests that our method
will also be capable of working on other deformable objects.

To demonstrate our confidence in the 3D face reconstruction method
presented in Chapter~\ref{chapter:face}, a website was
developed\footnote{http://cvl-demos.cs.nott.ac.uk/vrn}, allowing users
to upload an image of a face and be presented with an interactive 3D
model. The impact of which far exceeded any of our expectations. To
date, this website has reconstructed more than 1 million
faces. Feedback from users has been positive and the resultant models
have been used for a multitude of purposes; from comforting those who
have lost loved ones, to being used in music videos and assisting in
teaching school children about 3D printing. Additionally, code for our
3D face reconstruction method has been made
available\footnote{https://github.com/AaronJackson/vrn} under the MIT
license.

\section{Future Work}

In Chapter~\ref{chapter:face} we presented results which demonstrate
Volumetric Regression Networks as being a high performing,
state-of-the-art method for 3D face reconstruction. Despite these
results, 3D face reconstruction will continue to be an active research
area for quite some time. In this section, we will discuss some
potential directions for future work, specifically aimed at Volumetric
Regression Networks.

\subsection{Detail Refinement} % Detail

Volumetric Regression Networks require a significant amount of
training data to produce high quality results - this is a limitation
of our method. Furthermore, obtaining high quality \textit{and}
detailed data, containing blemishes and wrinkles, in a natural (or
\textit{in-the-wild} setting is difficult, if not impossible. As such,
there is a clear requirement for a method which is able to learn this
in a semi-supervised, if not entirely unsupervised manner.

\subsection{Texture Synthesis}

Our method for 3D face reconstruction does not attempt to produce any
kind of texture map or colour information, instead we focused on the
reconstruction aspect only. The online demo of this method, mentioned
earlier, does render faces with a texture. However, this is simply the
original image projected onto the 3D model using the $x,y$
coordinates. This occasionally results in strange artefacts around the
sides of the reconstruction. An interesting direction for future work
may be to improve this, by simultaneously regressing a warped texture
image which can wrap around the regressed 3D model.

\subsection{Fixed Correspondence}

One of the advantages of 3DMM based approaches to 3D face
reconstruction is that vertices have a fixed index across all meshes
(except for a few cases, such as~\cite{tran2018extreme}, where the
mesh is modified in a refinement step). As VRN produces a volumetric
representation, both the number of vertices and the order of them
varies, depending on the face or surface extraction algorithm
used. This somewhat hinders the ease-of-use for applications such as
applications, as vertices would have to be selected manually. There
are a number of potential solutions to this problem, such as
regressing a vertex map, in a similar fashion to aforementioned
texture synthesis. Alternatively, as a post-processing step,
resampling and ordering the vertices.

\subsection{Temporal Consistency}

As our method was designed to work from just a single image, no
consideration was really given to how well it would work over temporal
data. While we feel there performance is reasonably good on videos, it
is possible that the shape of the face might change, potentially
making it unsuitable for a variety of purposes. Hence, a nice addition
to our method would be to optionally accept features from several
previous frames. This would allow some temporal consistency to be
established, hopefully constraining the face shape over time.

\subsection{Performance and Speed}

The performance of our 3D reconstruction method is limited by what
makes it an effective method, the volumetric
representation. Unfortunately, 3D volumes are generally quite
large. The drawback of this is that it takes time to read and write
them to disk, as well as apply transformations for data
augmentation. However, there are perhaps several optimisations which
could be made to improve this. Examples might include a network
configuration which is using half precision floats, which would also
allow the models to be saved in half precision floats, instead of
converting them from bytes (which also takes time). There is plenty of
room for performance improvements towards the way they are stored on
disk - Run Length Encoding (RLE) could certainly be employed to
compact the contiguous blocks of ones or zeros into something
significantly faster to read.


%%% Local Variables:
%%% TeX-master: "../thesis"
%%% End:
