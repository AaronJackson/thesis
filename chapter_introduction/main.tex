\chapter{Introduction}

The landscape of Computer Vision has shifted dramatically over the
past decade. Prior to this, general purpose solutions to difficult
problems had low success rates on unconstrained images. This thesis
primarily focuses on 3D face reconstruction, which is to estimate the
3D geometry of a face from a one or more images. This problem
inherently demands reliability on ``in-the-wild'' images. Such images
pose many challenges, such as large variations in pose, expression and
lighting, as well as self occlusion. In this introductory chapter, we
will briefly discuss several general approaches to 3D face
reconstruction, including their advantages and disadvantages. The aim
of doing this is to show where, how and why, our own work fits into
challenging space.

% 3D Morphable Models
3D Morphable Models (3DMM) are perhaps the most popular approach to
estimating facial 3D geometry. These methods regress parameters, more
recently with Convolutional Neural Networks, for a face model [cite
things], which varies the shape, expression, pose and optionally
textural information. Such methods often work very reliably on
\textit{normal}, neutral, reasonably frontal faces. However, many 3DMM
methods fail dramatically under other conditions. As we show in
Section~\ref{chapter:face:sec:ablation}, our own method has a very
uniform and low error over all expressions, with only a small increase
in error on large poses (such as 80 degrees).

% Multiview

% Shape from Shading

% Depth Estimation

% Contributions. Unsure about whether to approach this as
% contributions or aims.

\section{Contributions}

The contributions of this thesis can be summarised as follows:

\begin{itemize}
\item % 3D face reconstruction
  We present a deeply learnt method for 3D face reconstruction, which
  can be trained directly in an end-to-end fashion. Our method
  reconstructions the full 3D facial structure, including parts which
  occluded. We also show that our method works from just a single
  images image without requiring accurate alignment or establishing a
  dense correspondence between all training samples.  This allows us
  to bypass the construction (during training) and fitting (during
  inference) of a 3D Morphable Model (3DMM).  We refer to this method
  as the \textbf{Volumetric Regression Network (VRN)}.

\item % We show that our method handles large variations in pose and
  % other difficult challenges.
  We show how a related task of facial landmark localisation can be
  incorporated into our method (also in an end-to-end fashion) to
  improve the reconstruction quality, particularly on faces of large
  pose and expression.

\item We evaluate our method on both constrained and heavily
  unconstrained images from the web, demonstrating that our method
  \textbf{outperforms all prior work} on single image 3D face
  reconstruction by a large margin.,

\item We show that our method works on other deformable objects, such
  as in the task of \textbf{human body reconstruction}, and that
  provided high quality training data is available, our method is able
  to both reconstruct difficult human poses and \textbf{fine details
    such as wrinkles} in clothing.

\end{itemize}

\section{Outline} This thesis is structured as follows: In
Chapter~\ref{chapter:background} . This includes Convolutional
Neural Networks (CNNs) and 3D volumes, along with how they are
generated via voxelisation. Following on from this is a literature
review of related work in Chapter~\ref{chapter:literature}. Our first
work, which focuses on the problem of 3D face reconstruction is
explained in Chapter~\ref{chapter:face}. An extension to this work,
performing 3D reconstruction of the full human body, is presented in
Chapter~\ref{chapter:human}. While seemingly unrelated to the primary
focus of the thesis, our preliminary work on semantic facial part
labelling, which heavily influenced our approach to 3D reconstruction,
is described in Chapter~\ref{chapter:seg}.




%%% Local Variables:
%%% TeX-master: "../thesis"
%%% End:
