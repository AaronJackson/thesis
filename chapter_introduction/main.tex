\chapter{Introduction}

The landscape of computer vision has shifted dramatically over the
past decade. Prior to this, general purpose solutions to difficult
problems had low success rates on unconstrained images. This is
especially true for the primary focus of this thesis: 3D face
reconstruction. In this chapter...

% which is to estimate the 3D geometry of a face from
% one or more images. The problem of 3D face reconstruction inherently
% demands reliability on ``in-the-wild'' images since, while there
% \textit{are} many applications, very few are able to accommodate such
% environmental constraints. These unconstrained images pose many
% difficult challenges, from large variations in pose and expression, to
% self occlusion and lighting variation.

% Applications of 3D face reconstruction span a wide range of areas. The
% more obvious examples being personalisation of computer games, or
% trying on accessories online, such as glasses. However, this area has
% many other applications, such as facial expression analysis for
% measuring emotional arousal, which can often be useful in
% psychological studies. Facial performance transfer is another
% potential application, widely used in creative industries, such as
% development of computer games and animated films where typically a
% large number facial point markers have to be attached to the actor's
% face. There are also medical applications, such as simulating the
% result of reconstructive surgery after an accident.

\section{Problem}

3D face reconstruction is the process of estimating the 3D geometry of
a person's face, but what exactly do we even mean by geometry?
Fundamentally, this geometry is a set of 3D vertices, connected by
edges, which defines 3D surface. This is more generally referred to as
a mesh. This mesh should capture the face's shape, expression and
pose. In a perfect world, it should also capture finer details, such
as facial hair, eye brows, eye lashes, wrinkles, spots, blemishes and
anything else we as humans can see on person's face. Ideally, it
should also capture facial accessories, such as glasses and jewellery.

Well, even with the laser precision of commercial 3D scanners,
satisfying all of these desires is currently impossible. What if, for
example, the person is wearing glasses? The beam will become distorted
or perhaps worse, blocked by the lenses, producing less than
satisfactory results, where the glasses have become \textit{one} with
the face. Despite this, 3D scanners are the go to solution when it
comes to obtaining a high resolution facial scan. However, doing so
comes with many limitations. Perhaps the worst limitation of all is
that the person being scanned has to sit very still while the scan is
being taken, surrounded by the machine that is taking the scan. This
feels very similar to photography in the early 1800's. Mercury vapour
aside - ``Stand still, please, there are five minutes to go until our
photo is exposed!''


% single image is the hardest kind.
% what even is the geometry lol

% occlusion, large pose, expression,

\section{blah}

In this introductory chapter, we will briefly discuss several general
approaches to 3D face reconstruction, including their advantages and
disadvantages. The aim of doing this is to show where, how and why,
our own work on 3D face reconstruction, fits into this challenging space.


% 3D Morphable Models
\paragraph{3D Morphable Models (3DMM)} are perhaps the most popular
approach to estimating facial 3D geometry. These methods regress
parameters for a pre-computed face
model~\cite{jourabloo2016large,huber2016multiresolution,zhu2016face,liu2016joint},
which varies the shape, expression, pose and optionally textural
information. These methods are increasingly using Convolutional Neural
Networks as a means to regress the parameters, and in general, work
very reliably on \textit{normal} faces. However, 3DMM based approaches
gradually perform worse as factors such as pose and expression
increase. We show in Section~\ref{chapter:face:sec:ablation} that our
method has a very uniform and low error over all expressions, with
only a small increase in error on large poses (such as around 80
degrees). Another disadvantage of 3DMM based approaches is that the
3DMM must be generated - this requires finding the dense
correspondence between all vertices of all samples, which also becomes
difficult as pose and expression changes. Finally, since these methods
are modelled, they are unable to directly produce finer details, such
as wrinkles. In an extension to our own work, we show
in Chapter~\ref{chapter:human} (where we perform 3D reconstruction of humans),
that our method is able to directly regress fine details in clothing.


\paragraph{Multiview and Photogrammetry} based methods require
multiple images from many different angles in order to estimate the 3D
geometry of a face (or other
object)~\cite{dou2018multi,dai2018coarse,Piotraschke_2016_CVPR,mayo20093d}. The
primary drawback of such methods is, of course, that multiple images
are not always available, especially under the aforementioned
environmental constraints associated with 3D face
reconstruction. While the final 3D reconstruction from such methods
can be of very high resolution and detailed, finding corresponding
features between each image is both memory and CPU intensive. Our own
method uses only a single RGB image, without any additional depth data
- from this single image, we regress the full 3D facial geometry.

\paragraph{Shape-from-Shading (SfS)} methods attempt to recover the 3D
geometry by analysing the shading and reflectance of a face relative
to a \textit{albedo} (mean) face. Similarly to 3DMM based methods, SfS
methods also tend to rely on a 3D model, inheriting many of the same
problems, such as finding a dense
correspondence~\cite{suwajanakorn2014total,jiang20183d}. Additionally,
many of these methods also require more than one image, which means
SfS methods often also inherit the problems associated with multiview
methods. Finally, more so than other methods, SfS methods struggle
with lighting - in general, there \textit{has} to be shadows, so over
or under illuminated faces may produce poor results when compared to
other methods. Our approach to 3D face reconstruction works under
almost all kinds of lighting, and we show this on a challenging
synthetic dataset.

\paragraph{Depth Estimation} is arguable more popular among problems
such as room reconstruction, however, there are several methods which
attempt to reconstruct the frontal part of the facial 3D geometry
using this technique~\cite{sun2011depth,sun2013depth}. It would be
fair to claim that these methods have declined in popularity. The
biggest flaw of depth estimation based approaches is that only the
visible parts can be reconstructed, unless other methods are employed
afterwards (such as 3DMM, again, inheriting many of the aforementioned
problems). While also working from a single image, our method is able
to hallucinate the non-visible parts of the face, including, to an a
certain extent, those which have been self occluded without having to
handle this cases on an individual basis.

\section{Contributions}

The contributions of this thesis can be summarised as follows:

\begin{itemize}
\item % 3D face reconstruction
  We present a deeply learnt method for 3D face reconstruction, which
  can be trained directly in an end-to-end fashion. Our method
  reconstructs the full 3D facial structure, including parts which are
  occluded. We also show that our method works from just a single
  images without requiring accurate alignment or establishing a dense
  correspondence between all training samples.  This allows us to
  bypass the construction (during training) and fitting (during
  inference) of a 3D Morphable Model (3DMM).  We refer to this method
  as the \textbf{Volumetric Regression Network (VRN)}.

\item We show how a related task of facial landmark localisation can
  be incorporated into our method (also in an end-to-end fashion) to
  improve the reconstruction quality, particularly on faces of large
  pose and expression.

\item We evaluate our method on both constrained and heavily
  unconstrained images from the web, demonstrating that our method
  \textbf{outperforms all prior work} on single image 3D face
  reconstruction by a large margin.

\item We show that our method works on other deformable objects, such
  as in the task of \textbf{human body reconstruction}, and that
  provided high quality training data is available, our method is able
  to both reconstruct difficult human poses and \textbf{fine details
    such as wrinkles} in clothing.
\end{itemize}

\section{Outline}

In \textbf{Chapter~\ref{chapter:background}}, an introduction to the
background material is provided. This includes a brief introduction to
gradient descent based optimisation, an overview of Convolutional
Neural Networks (CNNs) the problems batch normalisation attempts to
solve. This background discussion then moves onto volumetric
representations and algorithms, such as voxelisation, surface
extraction and some of the difficulties associated with volumetric
representations.

\textbf{Chapter~\ref{chapter:literature}} gives an overview of related
literature, starting with face alignment, which is often a
prerequisite to 3D face reconstruction methods. Continuing on, a
discussion of the various existing 3D face reconstruction is
given. The remainder of this chapter gives an overview of human pose
estimation, human body reconstruction and semantic part segmentation.

Our first work, focusing on landmark guided semantic part segmentation
of the human face is described in
\textbf{Chapter~\ref{chapter:seg}}. It is important to mention here
that this thesis does not focus on semantic segmentation, but this
work \textit{heavily} influenced our approach to the problem of 3D
reconstruction, since our method shifts the 3D reconstruction problem
into a volumetric segmentation problem.

\textbf{Chapter~\ref{chapter:face}} is the primary focus of our
thesis. This chapter describes our approach to 3D face reconstruction,
which includes details about the datasets uses, the volumetric
representation and the (small) amount of error it introduces, along
with our approach and network architectures. This chapter also
describes the numerous experiments carried out on the various
datasets, comparisons with other methods, and ablation studies on
pose, expression and influence of parameters used for guidance.

An extension to our work on 3D face reconstruction is described in
\textbf{Chapter~\ref{chapter:human}}, in which we show how our method
can be applied to on another category of deformable objects: the full
human body. We carry out experiments on two datasets, showing that our
approach can handle both difficult poses and finer details.

Finally, we conclude with a reflection and discussion of our work,
what impact it may have had and the numerous areas worthy of attention
as future work, in \textbf{Chapter~\ref{chapter:conclusion}}.

%%% Local Variables:
%%% TeX-master: "../thesis"
%%% End:
