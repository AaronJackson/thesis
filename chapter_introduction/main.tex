\chapter{Introduction}

The landscape of Computer Vision has shifted dramatically over the
past decade. Prior to this, general purpose solutions to difficult
problems had low success rates on unconstrained images. This thesis
primarily focuses on 3D face reconstruction, which is to estimate the
3D geometry of a face from a one or more images. This problem
inherently demands reliability on ``in-the-wild'' images. Such images
pose many challenges, such as large variations in pose, expression and
lighting, as well as self occlusion. In this introductory chapter, we
will briefly discuss several general approaches to 3D face
reconstruction, including their advantages and disadvantages. The aim
of doing this is to show where, how and why, our own work fits into
challenging space.

% 3D Morphable Models

% Multiview

% Shape from Shading

% Depth Estimation


\paragraph{Thesis Outline} This thesis is structured as follows: In
Chapter~\ref{chapter:background} and introduction to various topics
which our works depends on. This includes Convolutional Neural
Networks (CNNs) and 3D volumes and how they are generated. Following
on from this is a literature review of related work in
Chapter~\ref{chapter:literature}. Our first work, which focuses on the
problem of 3D face reconstruction is explained in
Chapter~\ref{chapter:face}. An extension to this work, performing 3D
reconstruction of the full human body is presented in
Chapter~\ref{chapter:human}. While seemingly unrelated to the main
focus of the thesis, our preliminary work on semantic facial part
labell, which heavily influenced our approach to 3D reconstruction, is
described in Chapter~\ref{chapter:seg}.




%%% Local Variables:
%%% TeX-master: "../thesis"
%%% End:
