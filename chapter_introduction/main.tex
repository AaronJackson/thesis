\chapter{Introduction}

The landscape of computer vision has shifted dramatically over the
past decade. Prior to this, general purpose solutions to difficult
problems had low success rates on unconstrained images. This is
especially true for the primary focus of this thesis: 3D face
reconstruction, which is to estimate the 3D geometry of a face from a
one or more images. The problem of 3D face reconstruction inherently
demands reliability on ``in-the-wild'' images since, while there
\textit{are} many applications, very few are able to accommodate such
environmental constraints. These unconstrained images pose many
difficult challenges, from large variations in pose and expression, to
self occlusion and lighting variation.

Applications of 3D face reconstruction span a wide range of areas. The
more obvious examples being personalisation of computer games, or
trying on accessories online, such as glasses. However, this area has
many other applications, such as facial expression analysis for
measuring emotional arousal, which can often be useful in
psychological studies. Facial performance transfer is another
potential application, widely used in creative industries, such as
development of computer games and animated films where typically a
large number facial point markers have to be attached to the actor's
face. There are also medical applications, such as simulating the
result of reconstructive surgery after an accident.

In this introductory chapter, we will briefly discuss several general
approaches to 3D face reconstruction, including their advantages and
disadvantages. The aim of doing this is to show where, how and why,
our own work on 3D face reconstructions, fits into challenging space.


% 3D Morphable Models
\paragraph{3D Morphable Models (3DMM)} are perhaps the most popular
approach to estimating facial 3D geometry. These methods regress
parameters for a pre-computed face model [cite things], which varies
the shape, expression, pose and optionally textural information. These
methods are increasingly using Convolutional Neural Networks as a
means to regress the parameters, and in general, work very reliably on
\textit{normal} faces. However, 3DMM based approaches gradually
perform worse as factors such as pose and expression increase. We show
in Section~\ref{chapter:face:sec:ablation} that our method has a very
uniform and low error over all expressions, with only a small increase
in error on large poses (such as around 80 degrees). Another
disadvantage of 3DMM based approaches is that the 3DMM must be
generated - this requires finding the dense correspondence between all
vertices of all samples, which also becomes difficult as pose and
expression changes. Finally, since these methods are modelled, they
are unable to directly produce finer details, such as wrinkles. In an
extension to our own work, we show in~\ref{chapter:human} (where we
perform 3D reconstruction of humans), that our method is able to
directly fine details in clothing.


\paragraph{Multiview and Photogrammetry} based methods require
multiple images from many different angles in order to estimate the 3D
geometry of a face (or other object). The primary drawback of such
methods, of course, is that multiple images are not always available,
especially under the aforementioned constraints associated with such a
problem. Our own method uses only a single RGB image, without any
additional depth data - from this single image, we regress the full 3D
facial geometry. While the final 3D reconstruction from such methods
can be of very high resolution and detail, finding corresponding
features between each image is both memory and CPU intensive. 

...


blah blah blah


...

% Multiview

% Shape from Shading

% Depth Estimation

% Contributions. Unsure about whether to approach this as
% contributions or aims.

\section{Contributions}

The contributions of this thesis can be summarised as follows:

\begin{itemize}
\item % 3D face reconstruction
  We present a deeply learnt method for 3D face reconstruction, which
  can be trained directly in an end-to-end fashion. Our method
  reconstructions the full 3D facial structure, including parts which
  occluded. We also show that our method works from just a single
  images image without requiring accurate alignment or establishing a
  dense correspondence between all training samples.  This allows us
  to bypass the construction (during training) and fitting (during
  inference) of a 3D Morphable Model (3DMM).  We refer to this method
  as the \textbf{Volumetric Regression Network (VRN)}.

\item % We show that our method handles large variations in pose and
  % other difficult challenges.
  We show how a related task of facial landmark localisation can be
  incorporated into our method (also in an end-to-end fashion) to
  improve the reconstruction quality, particularly on faces of large
  pose and expression.

\item We evaluate our method on both constrained and heavily
  unconstrained images from the web, demonstrating that our method
  \textbf{outperforms all prior work} on single image 3D face
  reconstruction by a large margin.,

\item We show that our method works on other deformable objects, such
  as in the task of \textbf{human body reconstruction}, and that
  provided high quality training data is available, our method is able
  to both reconstruct difficult human poses and \textbf{fine details
    such as wrinkles} in clothing.

\end{itemize}

\section{Outline} This thesis is structured as follows: In
Chapter~\ref{chapter:background} . This includes Convolutional
Neural Networks (CNNs) and 3D volumes, along with how they are
generated via voxelisation. Following on from this is a literature
review of related work in Chapter~\ref{chapter:literature}. Our first
work, which focuses on the problem of 3D face reconstruction is
explained in Chapter~\ref{chapter:face}. An extension to this work,
performing 3D reconstruction of the full human body, is presented in
Chapter~\ref{chapter:human}. While seemingly unrelated to the primary
focus of the thesis, our preliminary work on semantic facial part
labelling, which heavily influenced our approach to 3D reconstruction,
is described in Chapter~\ref{chapter:seg}.




%%% Local Variables:
%%% TeX-master: "../thesis"
%%% End:
